\documentclass[pdftex,11pt,a4paper]{article}

%Paquetes importantes
\usepackage[left=2cm,top=1cm,right=2cm,bottom=2cm]{geometry} %Formato
\usepackage[english]{babel} %Para escribir en español
% \decimalpoint
\usepackage[utf8]{inputenc} %Acentos y demás
\usepackage{amsmath} %Simbolos y cosas bonitas
\usepackage{amsfonts} %Simbolos y cosas bonitas
\usepackage{array} %Tablas bonitas
\usepackage{listings} % Insertar Código
\usepackage[pdftex]{graphicx} % Imagenes
\usepackage{hyperref} %Hipervinculos
\usepackage{pdflscape} %Poner algunas páginas en Horizontal \begin{landscape}
\usepackage{multicol}  % Latex a Dos Columnas
\usepackage{float} % Poner mejor las imágenes.

\setlength\parindent{0pt} % Quitar las sangías
% \setlength{\intextsep}{1pt plus 1pt minus 1pt} % Pámetro de texto entre figura e imágen.

\title{Linear Algebra}
\author{Paolo Luciano Rivera}
\date{2021}

%%%%%%%%%%%%%%%%%%%%%%%%%%%%%%%%%%%%%%%%%%%%%%%%%%%%%%%

\newcommand{\R}{\mathbb{R}}
\newcommand{\B}{\mathbf{B}}

\newcommand{\zero}{\mathbf{0}}
\newcommand{\ihat}{\hat{\textbf{\i}}}
\newcommand{\jhat}{\hat{\textbf{\j}}}
\newcommand{\xn}{\mathbf{x}}
\newcommand{\yn}{\mathbf{y}}
\newcommand{\un}{\mathbf{u}}
\newcommand{\vn}{\mathbf{v}}


\begin{document}
\maketitle

\section{Definitions}
\subsection{Vectors and Vector Spaces}
\begin{itemize}
	\item The \textit{span} of $\un$ and $\yn$ is the set of all \textit{linear combinations}:
	$$a \un + b \vn$$ 
	\item When $\un$ is a linear combination of the others basis vectors, then it's \textit{linearly dependent}
 	\item The basis $\B$ of a \textit{vector space} $V$ is a set of \textit{linearly independent} vectors that \textit{span} the full space
\end{itemize}

\section{Linear Algebra Intuition}
Notes from the \href{https://www.youtube.com/watch?v=fNk_zzaMoSs&list=PLZHQObOWTQDPD3MizzM2xVFitgF8hE_ab}{3B1B video series.}
\subsection{Vectors and Vector Spaces}
\begin{itemize}
	\item Vectors $\xn = (x_1, \ldots, x_n)$ can be viewed interchangeably as \textbf{ordered lists} or as objects with length and direction, i.e. an arrow
	\item Vectors are points in L.A., sometimes it's easier to think of them as arrows, sometimes as points (when you have many), sometimes as lists (when you're programming)
	\item Vectors live in a \textbf{vector space} $V$ that has certain properties 
	\item in Linear Algebra (L.A.) vectors always start at the origin $\zero = (0,\ldots,0)$
	\item Vectors are almost always represented as column vectors
	\item Vectors have a sum $+$ and a scalar multiplication $\cdot$
	\item Vector sum can be viewed as a translation to the vector space, similarly to how real numbers translate the $\R$
	\item Scalar multiplication can be viewed as \textit{scaling} the vector by the magnitude of the scalar.
	\item Any vector $\xn$ can be viewed as the \textit{linear combination} of the basis vectors (on a given space). e.g. for $\R^2$
	$$\xn = 3\ihat -2\jhat = (3,-2)$$
	\item When you think of a vector space, you have to think about the basis of it $\B$, in the previous example $\ihat$ and $\jhat$ \textit{span} $\R^2$
	
\end{itemize}


\section{Useful Properties}
\subsection{Matrices}
\begin{itemize}
	\item \textbf{$2x2$ matrices}
	$A = \begin{bmatrix}
	a&b\\c&d
	\end{bmatrix} $
	\begin{itemize}
		\item \textbf{Inverse} If $\det(A) = ad -cd \neq 0 \Rightarrow A^{-1} = \dfrac{1}{ad-cd}	\begin{bmatrix}
			d & -b \\
			-c & a
		\end{bmatrix}$
	\end{itemize}
\end{itemize}
\end{document}