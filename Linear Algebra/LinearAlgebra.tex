\documentclass[pdftex,11pt,a4paper]{article}

%Paquetes importantes
\usepackage[left=2cm,top=1cm,right=2cm,bottom=2cm]{geometry} %Formato
% \usepackage[english]{babel} %Para escribir en español
% \decimalpoint
\usepackage[utf8]{inputenc} %Acentos y demás
\usepackage{amsmath} %Simbolos y cosas bonitas
\usepackage{amsfonts} %Simbolos y cosas bonitas
\usepackage{array} %Tablas bonitas
% \usepackage{listings} % Insertar Código
\usepackage[pdftex]{graphicx} % Imagenes
\usepackage{hyperref} %Hipervinculos
% \usepackage{pdflscape} %Poner algunas páginas en Horizontal \begin{landscape}
\usepackage{multicol}  % Latex a Dos Columnas
\usepackage{float} % Poner mejor las imágenes.

\setlength\parindent{0pt} % Quitar las sangías
% \setlength{\intextsep}{1pt plus 1pt minus 1pt} % Pámetro de texto entre figura e imágen.

\title{Linear Algebra}
\author{Paolo Luciano Rivera}
\date{2021}

%%%%%%%%%%%%%%%%%%%%%%%%%%%%%%%%%%%%%%%%%%%%%%%%%%%%%%%

\newcommand{\R}{\mathbb{R}}
\newcommand{\B}{\mathbf{B}}

\newcommand{\zero}{\mathbf{0}}
\newcommand{\one}{\mathbf{1}}
\newcommand{\ihat}{\hat{\textbf{\i}}}
\newcommand{\jhat}{\hat{\textbf{\j}}}
\newcommand{\xn}{\mathbf{x}}
\newcommand{\yn}{\mathbf{y}}
\newcommand{\un}{\mathbf{u}}
\newcommand{\vn}{\mathbf{v}}
\newcommand{\wn}{\mathbf{w}}
\begin{document}
\maketitle

\section{Definitions}
\subsection{Vectors and Vector Spaces}
\begin{itemize}
	\item The set $V$ is called a \textit{vector space} over field $F$ when the \textit{vector addition} and \textit{scalar multiplications} operations satisfy the following properties (for all $\un, \vn, \wn \in V$ and $a, b \in F$)
	\begin{itemize}
		\item \textbf{Closure property for addition}: $\un + \vn \in V$
		\item \textbf{Associativity of addition}: $\un + (\vn + \wn) = (\un + \vn) + \wn$
		\item \textbf{Commutativity of addition}: $\un + \vn = \vn + \un$
		\item \textbf{Zero of addition}: There is an element $\zero \in V$ such that $\un + \zero = \un$
		\item \textbf{Inverse element of addition}: For every $\un \in V$ there exists an element $-\un \in V$, called the additive inverse of $\un$ such that $\un + (-\un) = \zero$
		\item \textbf{Closure of multiplication}: $a \un \in V$ for all $a \in F$
		\item \textbf{Compatibility of multiplication with field}: $(ab)\un = a(b\un)$
		\item \textbf{Identity element of multiplication}: There is an element $\one \in V$ such that $\one \un = \un$
		\item \textbf{Distributivity of multiplication with respect to addition}: $a(\un + \vn) = a\un + a\vn$
		\item \textbf{Distributivity of multiplication with respect to field addition}: $(a + b)\un = a\un + b\un$
	\end{itemize}
	
	\item A \textit{linear combination} is $\wn = a\un + b\vn$
	\item The \textit{span} of $\un$ and $\yn$ is the set of \textbf{all} \textit{linear combinations} ($a, b \in \mathbb{R}$):
	$$a \un + b \vn$$  
	\item When $\un$ is a linear combination of the others basis vectors, then it's \textit{linearly dependent}
	\item If $\un \neq a\vn + b \wn$ for all values of $a, b$ then $\un$ is linearly independent
 	\item The basis $\B$ of a \textit{vector space} $V$ is a set of \textit{linearly independent} vectors that \textit{span} the full space
\end{itemize}

\subsection{Linear Transformations and Matrices}

\section{Linear Algebra Intuition}
Notes from the \href{https://www.youtube.com/watch?v=fNk_zzaMoSs&list=PLZHQObOWTQDPD3MizzM2xVFitgF8hE_ab}{3B1B video series.}
\subsection{Vectors and Vector Spaces (Chap 1 \& 2)}
\begin{itemize}
	\item Vectors $\xn = (x_1, \ldots, x_n)$ can be viewed interchangeably as \textbf{ordered lists} or as objects with length and direction, i.e. an arrow
	\item Vectors are points in L.A., sometimes it's easier to think of them as arrows, sometimes as points (when you have many), sometimes as lists (when you're programming)
	\item Vectors live in a \textbf{vector space} $V$ that has certain properties 
	\item In Linear Algebra (L.A.) vectors always start at the origin $\zero = (0,\ldots,0)$
	\item Vectors are almost always represented as column vectors
	\item Vectors have a sum $+$ and a scalar multiplication $\cdot$
	\item Vector sum can be viewed as a translation to the vector space, similarly to how real numbers translate the $\R$
	\item Scalar multiplication can be viewed as \textit{scaling} the vector by the magnitude of the scalar
	\item Any vector $\xn$ can be viewed as the \textit{linear combination} of the basis vectors (on a given space). e.g. for $\R^2$
	$$\xn = 3\ihat -2\jhat = (3,-2)$$
	\item When you think of a vector space, you have to think about the basis of it $\B$, in the previous example $\ihat$ and $\jhat$ \textit{span} $\R^2$. This means that any time you write a vector, it implicitly depends on the choice o f basis you're using!
	\item In higher dimensions, the \textit{span} is more important because it creates \textit{subspaces}. e.g. $\mathbb{R}^2$ is a subspace of $\mathbb{R}^3$, i.e. a plane that cuts through 3D space
\end{itemize}

\subsection{Linear Transformations and Matrices (Chap 3 \& 4)}
\begin{itemize}
	\item A \textit{linear transformation} can be seen as a function that takes an input and returns one of the same class. In this case, vectors in a given vector space. The transformation however, implies \textit{movement}, in this case, the vector space
	\item In the case, given a matrix $A$, its multiplication scales all the space itself! However, it does satisfy certain special properties: a) all lines remain lines and b) the origin remains fixed. I.e. grid lines remain parallel and evenly spaced. e.g. rotations around the origin, translations, etc. 
	\item An easy way to think about a linear transformation (matrix) $A$ is to take each column and associate it with each basis vectors e.g. $\ihat$ and $\jhat$ for 2D. Each basis vector will land on each column of $A$ fully specifying the transformation
\end{itemize}



\section{Useful Properties}
\subsection{Matrices}
\begin{itemize}
	\item \textbf{$2x2$ matrices}
	$A = \begin{bmatrix}
	a&b\\c&d
	\end{bmatrix} $
	\begin{itemize}
		\item \textbf{Inverse} If $\det(A) = ad -cd \neq 0 \Rightarrow A^{-1} = \dfrac{1}{ad-cd}	\begin{bmatrix}
			d & -b \\
			-c & a
		\end{bmatrix}$
	\end{itemize}
\end{itemize}
\end{document}