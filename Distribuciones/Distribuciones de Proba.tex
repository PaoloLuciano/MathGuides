\documentclass[11pt, a4paper]{article}
\usepackage[left=.5cm,top=.1cm,right=.5cm,bottom=.5cm]{geometry}
\usepackage{pdflscape}
\usepackage[spanish]{babel}
\usepackage[utf8]{inputenc}
\usepackage{amsmath}
\usepackage{amsfonts}
\usepackage{array}
\pagenumbering{gobble}

\DeclareMathOperator{\Var}{Var}
\DeclareMathOperator{\E}{\mathbb{E}}
\DeclareMathOperator{\Po}{Po}
\DeclareMathOperator{\Bi}{Bi}
\DeclareMathOperator{\Be}{Be}
\DeclareMathOperator{\Geo}{Geo}
\DeclareMathOperator{\BN}{BN}
\DeclareMathOperator{\U}{U}
\DeclareMathOperator{\N}{N}
\DeclareMathOperator{\Exp}{Exp}
\DeclareMathOperator{\Beta}{B}

\begin{document}
\begin{landscape}
{\LARGE\textbf{Distribuciones Discretas}}\\
\\
\renewcommand{\arraystretch}{2}
\begin{tabular}{l|c|c|c|c|c| m{7cm}}

%HEADER
	\textbf{Distribución} & Función de Densidad & 
	\begin{tabular}{@{}		c@{}}
		Media \\ $\E [X]$
	\end{tabular} & 
	\begin{tabular}{@{}c@{}}
		Varianza \\ $\Var [X]$
	\end{tabular} &
	\begin{tabular}{@{}c@{}}
		Función Generadora \\
		de Momento
	\end{tabular}& 
	$ \E[X^a]$ & 
	Notas/Uso \\
	\hline
	
%BERNOULI
\begin{tabular}{@{}l@{}}
		\textbf{Bernoulli} \\
		 $X \sim \Be(p)$
	\end{tabular} & 
	\begin{tabular}{@{}c@{}}
		$f(x)= p^x (1-p)^{1-x}$ \\
		$ x =0,1; \quad 0 < p < 1 $
	\end{tabular} &
	$p$ & 
	$p(1-p)$ &
	$pe^t + (1-p)$ &
	$\forall, a \in \mathbb{N} \quad \E[X^a]=p$ &
	\begin{tabular}{l}
	$p:=$ proba. de éxito. 
	\end{tabular} \\
	\hline 

%BINOM
	\begin{tabular}{@{}l@{}}
		\textbf{Binomial} \\
		 $X \sim \Bi(n,p)$
	\end{tabular} & 
	\begin{tabular}{@{}c@{}}
		$f(x)= \dbinom{n}{x} p^x q^{n-x}$ \\
		$ x =0,1,...,n; \quad 0 < p < 1 $
	\end{tabular} &
	$np$ & 
	$npq$ &
	$(pe^t+q)^n $ &
	
	\renewcommand{\arraystretch}{1.3}
	\begin{tabular}{@{}c@{}} 
		$\E(X^a)=np\E[(Y+1)^{a-1}] $ \\
		Con $Y$: v.a. binomial con \\
		parámetros $n-1,p$
	\end{tabular} & 
	
	\renewcommand{\arraystretch}{1.4}
	\begin{tabular}{l}
		Número de éxitos en $n$ ensayos.\\	
		$n:=$ número de ensayos. \\
		$p:=$ proba. de éxito. \\
		$q:=(1-p):=$ proba de fracaso.	
	\end{tabular}
\\ \hline 

%GEOM
	\begin{tabular}{@{}l@{}}
		\textbf{Geométrica} \\
		$X \sim  \Geo(p)$
	\end{tabular}&
	\renewcommand{\arraystretch}{1.5}
	\begin{tabular}{@{}c@{}}
		$f(x)=q^xp $ \\
		$ x =0,1,2...; \quad 0 \leq p \leq 1$\\
		Puede ser $q^{y-1}; \quad y=1,2...$
	\end{tabular}&
	$\dfrac{1}{p}$&
	$\dfrac{1-p}{p^2}$&
	$\dfrac{pe^t}{1-qe^t}$ &
	&
	\renewcommand{\arraystretch}{1.5}
	\begin{tabular}{m{7cm}}
		Número de ensayos en el que ocurre el primer éxito. \\
		$p:=$ proba. de éxito. \\
		$q:=(1-p):= $ proba de fracaso. 
	\end{tabular}
\\\hline

%POISS
	\begin{tabular}{@{}l@{}}
		\textbf{Poisson} \\
		 $X \sim$ $\Po(\lambda)$
	\end{tabular} & 
	\begin{tabular}{@{}c@{}} 
		$f(x)= \dfrac{\lambda^x e^{-\lambda}}{x!}$ \\
		$x=0,1,2,... ; \quad \lambda > 0 $
	\end{tabular}& 
	$\lambda$&
	$\lambda$&
	$e^{\lambda(e^t-1)}$&
	&
	\renewcommand{\arraystretch}{1.7}
	\begin{tabular}{m{7cm}}
		Número de eventos que ocurren en un intervalo de tiempo (región) con un promedio 			conocido. Binomial con $n$ grande. $\lambda = np:=$ promedio 	conocido.
	\end{tabular}
\\ \hline

% BINOM NEGATIVA
	\renewcommand{\arraystretch}{1.2}
	\begin{tabular}{@{}c@{}}
		\textbf{Binomial}\\
		\textbf{Negativa}\\
		$X \sim \BN(r)$
	\end{tabular} &
	\begin{tabular}{@{}c@{}}
		$f(x)= \dbinom{x-1}{r-1}p^rq^{x-r}$ \\
		$ x =r,r+1,... $
	\end{tabular} &
	$\dfrac{r}{p}$&
	$\dfrac{rq}{p^2}$&
	$\left(\dfrac{p}{1-qe^t}\right)^r$&
	\renewcommand{\arraystretch}{1.7}
	\begin{tabular}{@{}c@{}}
		$\dfrac{r}{p}\E[(Y-1)^{a-1}]$ \\
		$Y$ v.a. binomial negativa \\
		con parámetros r+1,p\\
	\end{tabular} &
	\renewcommand{\arraystretch}{1.5}
	\begin{tabular}{m{7cm}}
	Número de evento en el que ocurre el \, \, r-ésimo éxito.\\ 
	$r:=$ r-ésimo éxito.\\
	\end{tabular}

	\end{tabular}	
\end{landscape}

%CONTINUAS
\begin{landscape}
{\LARGE\textbf{Distribuciones Continuas}}\\
\\
\renewcommand{\arraystretch}{2.1}
\begin{tabular}{l|c|c|c|c|c| m{7cm}}
	\textbf{Distribución} &
	Función de Densidad &
	\renewcommand{\arraystretch}{1.1}
	\begin{tabular}{@{}c@{}}
		Media \\
		$\E[X]=\mu$
	\end{tabular} &
	\renewcommand{\arraystretch}{1.1} 
	\begin{tabular}{@{}c@{}}
		Varianza \\
		$V[X]=\sigma^2$
	\end{tabular} &
	\renewcommand{\arraystretch}{1.1}
	\begin{tabular}{@{}c@{}}
		Función Generadora \\
		de Momento
	\end{tabular}& 
	$\E[X^a]$ & 
	Notas/Uso 
\\ \hline

%UNIFORME
	\renewcommand{\arraystretch}{1.9}
	\begin{tabular}{@{}l@{}}
	\textbf{Uniforme} \\
	$X \sim \U(a,b)$	
	\end{tabular} & 
	\begin{tabular}{@{}c@{}}
	$f(x)=\dfrac{1}{b-a}$\\
	$a \leq y \leq b$
	\end{tabular} &
	$\dfrac{a+b}{2}$ & 
	$\dfrac{(b-a)^2}{12}$ &
	$\dfrac{e^{tb}-e^{ta}}{t(b-a)}$ &	
	\begin{tabular}{@{}c@{}} 
	\end{tabular}& 
	Poisson en un intervalo $[a,b]$ en el que sólo pasa 1 evento. 			Misma probabilidad en 		cada punto del intervalo
	\\ \hline

%NORMAL
	\begin{tabular}{@{}l@{}}
		\textbf{Normal} \\
		$X \sim \N(\mu,\sigma)$
	\end{tabular} & 
	\begin{tabular}{@{}c@{}}
		$f(x)=\dfrac{1}{\sqrt{2\pi}\sigma}e^{-\dfrac{\left( 					x-\mu\right)^2}{2\sigma^2}}$
	\end{tabular} &
	$\mu$ & 
	$\sigma^2$ &
	$e^{\mu t + (t^2\sigma^2)/2}$ &
	$
	\begin{cases} 
		0 & \text{si } a \text{ es par}\\
		\sigma^a(a-1)!! & \text{O.C.}			
	\end{cases}$ & 
	Muchos éxitos idénticos e independientes. Se busca 						estandarizar la función usando: $Z=\N(0,1)=\frac{Y-\mu}{\sigma}.$\\ 
	\hline
	
%GAMMA
	\begin{tabular}{@{}l@{}}
		\textbf{Gamma} \\
		 $X \sim \Gamma(\alpha,\beta)$
	\end{tabular} & 
	\begin{tabular}{@{}c@{}}
		$f(x)=\dfrac{\beta^\alpha}{\Gamma(\alpha)}								x^{\alpha-1}e^{-x\beta}$\\
		$ 0 < x < \infty $
	\end{tabular} &
	$\alpha/\beta$ & 
	$\alpha/\beta^2$ &
	$(1-\beta t)^{-\alpha}$ &
	\begin{tabular}{@{}c@{}}
		$\E(Y^a)=\dfrac{\beta^\alpha\Gamma(\alpha+a)}{\Gamma(\alpha)}$
	\end{tabular}&
	\renewcommand{\arraystretch}{1.2} 
	\begin{tabular}{@{}l@{}}
		Tiempos. $\alpha>0$ y $\beta>0$.\\
		Prop. $\int^{\infty}_{0}y^{\alpha-1}e^{-y/\beta}dy=\beta^				\alpha\Gamma(\alpha)$\\
		Propiedades de Gamma: $\Gamma(1)=1$\\ 
		$\Gamma(\alpha)=(\alpha-1)\Gamma(\alpha-1)$\\
		$\Gamma(n)=(n-1)!$ si $(n-1) \in \mathbb{Z}$
	\end{tabular}\\
	\hline
	
%EXPONENCIAL
	\renewcommand{\arraystretch}{2} 
	\begin{tabular}{@{}l@{}}
		\textbf{Exponencial} \\
 		$X \sim \Exp(\beta)$
	\end{tabular} & 
	\begin{tabular}{@{}c@{}}
		$f(x)=\beta e^{-x\beta}$\\
		$ 0 < x < \infty $
	\end{tabular} &
	$1/\beta$ & 
	$1/\beta^2$ &
	$(1-\beta t)^{-1}$ &
	\begin{tabular}{@{}c@{}}
		Misma que Gamma. 
	\end{tabular}&
	\renewcommand{\arraystretch}{1.2} 
	\begin{tabular}{@{}m{6cm}}
		Parametro de intensidad $\beta$\\
		Parametro de media $\lambda = 1/\beta$\\
		Caso especial de la Gamma cuando $\alpha=1$.\\
		Geométrica en caso continuo.\\ 
		Propiedad de falta de memoria: Sean $a,b>0 \Rightarrow $\\
		$ P(Y>a+b|Y>a)=P(Y>b) $\\ 
	\end{tabular}\\
	\hline

%JI-CUADRADA
	\renewcommand{\arraystretch}{2}
	\begin{tabular}{@{}l@{}}
		\textbf{Ji-Cuadrada} \\
		$X \sim \chi^2(v)$
	\end{tabular} & 
	\begin{tabular}{@{}c@{}}
		$f(x)=\dfrac{x^{(v/2)-1}e^{-x/2}}{2^{v/2}\Gamma(v/2)}$\\
		$ 0 < x < \infty $
	\end{tabular} &
	$v$ & 
	$2v$ &
	$(1-2t)^{-v/2}$ &
	\begin{tabular}{@{}c@{}}
		Misma que Gamma. 
	\end{tabular}& 
	\renewcommand{\arraystretch}{1.2} 
	\begin{tabular}{@{}m{6cm}}
	Caso especial de la Gamma \\
	cuando $\alpha=v/2;$ $\beta=2;$\\
	$v:=$ grados de libertad
	\end{tabular} \\
	\hline

%BETA
	\renewcommand{\arraystretch}{2} 
	\begin{tabular}{@{}l@{}}
		\textbf{Beta} \\
		$X \sim \Beta(\alpha,\beta)$
	\end{tabular} &
	\begin{tabular}{@{}c@{}}
		$f(x)=\dfrac{x^{\alpha-1}(1-x)^{\beta-1}}						{\beta(\alpha,		\beta)}$\\
		$0 \leq x \leq 1$
	\end{tabular} &
	$\dfrac{\alpha}{\alpha+\beta}$ & 
	$\frac{\alpha\beta}{(\alpha+\beta)^2(\alpha+\beta+1)}$ &
	No existe & 
	& 
	\renewcommand{\arraystretch}{1.2} 
	\begin{tabular}{@{}m{6cm}}
		Proporciones y Porcentajes\\
		$\alpha,\beta > 0$. Si $\alpha,\beta \in \mathbb{Z}$ se 				relaciona con la binomial donde $n=\alpha+\beta-1$ y el 				contador comienza en $\alpha$.\\
		$\beta(\alpha,\beta)=\dfrac{\Gamma(\alpha)\Gamma(\beta)}				{\Gamma(\alpha+\beta)} $ 
	\end{tabular}\\
\end{tabular}
\end{landscape}
\end{document}